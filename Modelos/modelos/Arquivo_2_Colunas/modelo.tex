\documentclass[twocolumn]{article}
\usepackage[portuguese]{babel}
\usepackage[utf8]{inputenc}
\usepackage{amsmath}
\usepackage{subcaption}
\usepackage{mathtools}
\usepackage{graphicx}
\usepackage{color}
\usepackage{authblk}
\usepackage[colorlinks,citecolor=red,urlcolor=blue,bookmarks=false,hypertexnames=true]{hyperref}
\usepackage{geometry}

% Tamanho das margens:
\geometry{
	a4paper,
	total={170mm,257mm},
	left=30mm,
	top=20mm,
}

\usepackage[alf, abnt-etal-list=0, abnt-emphasize=bf,abnt-last-names=bibtex, abnt-etal-text=it, abnt-etal-cite=2]{abntex2cite}

\title{Título}
\author{Leandro Vieira dos Santos}
\affil{EREM Regina Pacis\\Palmerina-PE}

\usepackage{lipsum}

\begin{document}

\maketitle        

\begin{abstract}
\lipsum[1]
\end{abstract}

\section{Introdução}

\lipsum[1-2]

\section{Metodologia}

\lipsum[3-6]

\subsection{Conexão em Estrela}\label{sec_estrela}
Foi montada a partida do motor em estrela, conforme diagramas das Figuras~\ref{comando} e \ref{estrela}.

\begin{figure}[!htb]
	\centering\includegraphics[width=\columnwidth]{diagrama_estrela}\\
	\caption{Circuito de comando. }\label{comando}
\end{figure}

\begin{figure}[!htb]
	\centering\includegraphics[width=\columnwidth]{diagrama_estrela}\\
	\caption{Conexão do motor trifásico em estrela. }\label{estrela}
\end{figure}

\subsection{Conexão em Triângulo}
A Figura~\ref{triangulo} apresenta o diagrama de força da conexão em triângulo. O diagrama de comando utilizado é o mesmo da conexão em estrela, ilustrado na Figura~\ref{comando}.

\begin{figure}[!htb]
	\centering\includegraphics[width=\columnwidth]{diagrama_estrela}\\
	\caption{Conexão do motor trifásico em estrela. }\label{triangulo}
\end{figure}

\section{Resultados e Discussão}

Esta seção deve apresentar o que o aluno ou grupo obteve ao seguir o roteiro ou resolver a situação-problema proposta. Não é necessário repetir informações da metodologia. Quando necessário, pode-se fazer referência, ao que foi exposto no texto da Seção~\ref{sec_estrela}, por exemplo.

Nesta seção, espera-se que sejam apresentadas descrições do funcionamento do circuito, medições, comparação entre os métodos utilizados, além de justificativas para o que for observado. Para todas as medições, deve ser informado em qual circuito foram colhidas, informando unidades e os equipamentos de medida utilizados para sua aferição. 

\section{Conclusão}
Esta seção deve ser curta, resumindo a aprendizagem obtida na aula. Nenhuma nova informação deve ser apresentada. Devem ser feitas menções aos objetivos da aula, aos principais resultados e os conceitos apreendidos com estes. Pode ser feito na forma de itens:

\begin{itemize}
	\item Partida direta de motor de indução trifásico pode ser feita em estrela ou triângulo
	\item As correntes nominais e de partidas são maiores em triângulo.
	\item Cada conexão é apropriada para um nível de tensão da rede.
	\item Para o motor utilizado, a conexão em estrela serve para redes de 220~V ou para auxiliar a partida; enquanto que a conexão em triângulo serve para redes de 127~V apenas.
\end{itemize}


%%%%%%%%%%%%%%%%%%%%%%%%%%5
% BIBLIOGRAFIA 
% Estilo de bibliografia ABNT. Se não tiver instalado, mude para plain ou ieeetr

%\bibliographystyle{plain} % Inclua isso se não tiver ABNTEX instalado
\bibliography{refs}
\begin{thebibliography}{refs}
\bibitem{}

\end{thebibliography}
\end{document}